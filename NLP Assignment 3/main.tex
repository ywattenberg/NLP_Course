\documentclass[a4paper,12pt]{ETHexercise}
\usepackage{bbm}

\textheight24cm
\topmargin-2.5cm
\oddsidemargin+0cm
\textwidth16.3cm 

\usepackage{amsmath}
\usepackage{amsfonts}
\usepackage{amssymb}
\usepackage{color}
\usepackage[latin1]{inputenc}
\usepackage{ifthen}
\usepackage{enumerate}
\usepackage{lastpage}
\usepackage{graphicx}
%\usepackage{subfigure}
\usepackage{subcaption}
\usepackage{bm}
\usepackage{tikz}
\usepackage{pgfplots}
\usepackage{float}
\usepackage{nicefrac}
\usepackage{epsfig}
\usepackage{etoolbox}
\usepackage{framed}
\usepackage{enumitem}
\usepackage{hyperref}
\usepackage{datetime}
\usepackage{url}
%\usepackage{algorithm}
\usepackage{svg}
\usepackage[ruled,vlined]{algorithm2e}
\usepackage{setspace}
\usepackage[htt]{hyphenat}
\usepackage{qtree}
\usepackage{forest}
\usepackage{inconsolata} % for texttt{}


\usepackage[noend]{algpseudocode}
\algnewcommand{\parState}[1]{\State%
    \parbox[t]{\dimexpr\linewidth-\algmargin}{\strut\hangindent=\algorithmicindent \hangafter=1 #1\strut}}

\algrenewcommand\algorithmicindent{1.0em}%
\renewcommand\algorithmicdo{:}
\renewcommand\algorithmicthen{:}
\input{math_commands}

%\newcommand{\linetofill}{\ \\\hphantom{\hspace{0mm}}\hrulefill\ \\}
\newcommand{\linetofill}{\ \\\hphantom{\hspace{0mm}}\dotfill\ \\}
\newcommand{\vlinetofill}[1]{\\\rule{#1 mm}{0.1mm}\ \\}
\newcommand{\cbox}{
  \setlength{\unitlength}{1pt}
  \begin{picture}(10,10)
    \put(0,0){\line(1,0){8}} \put(0,8){\line(1,0){8}}
    \put(0,0){\line(0,1){8}} \put(8,0){\line(0,1){8}}
  \end{picture}
}
\newcommand{\bigcbox}{
  \setlength{\unitlength}{3pt}
  \begin{picture}(10,10)(1,1)
    \put(0,0){\line(1,0){8}} \put(0,8){\line(1,0){8}}
    \put(0,0){\line(0,1){8}} \put(8,0){\line(0,1){8}}
  \end{picture}
}
% points and boxes on the right
\newcommand{\points}[1]{
\ \\[-5mm]
\hphantom{\ }\hfill\textbf{#1 pts \bigcbox \hspace{-6mm}}
\vspace{5mm}
}
% multiple choice checkboxes
%\newcommand{\boxt}{\hspace*{2em} $[~~]$ \textsf{True}}
%\newcommand{\boxf}{\hspace*{2em} $[~~]$ \textsf{False}}
\newcommand{\boxt}{\hspace*{2em} $\square$ \textsf{True}}
\newcommand{\boxf}{\hspace*{2em} $\square$ \textsf{False}}
\newcommand{\justif}{\textsf{Justification}:  \rule{0ex}{2em}\dotfill\\\rule{0ex}{2em}\dotfill}
\newcommand{\checkboxWithJustification}{\boxt \boxf\\ \justif\\}
\newcommand{\checkbox}{\boxt \boxf}
\usepackage{tikz}
\usepackage{xcolor}
\usepackage{amssymb}
\usepackage{amsmath}
\usepackage{amsthm}
\usepackage{pgfplots}
\pgfmathdeclarefunction{gauss}{2}{%
  \pgfmathparse{1/(#2*sqrt(2*pi))*exp(-((x-#1)^2)/(2*#2^2))}%
}
\renewcommand{\S}{{\cal S}}
\usepackage{pgfplots}
    \pgfplotsset{compat=newest}
    \pgfplotsset{small, every non boxed x axis/.append style={x axis line style=-},
every non boxed y axis/.append style={y axis line style=-}}

\newcommand{\timestamp}{\ddmmyyyydate\today \,\,- \currenttime h}
%%% Local Variables:
%%% mode: latex
%%% TeX-master: "exam"
%%% End:

% Numbers
\usepackage[group-separator={,}]{siunitx}

\usepackage{cleveref}
\crefname{section}{\S}{\S\S}
\Crefname{section}{\S}{\S\S}
\crefname{table}{Tab.}{}
\crefname{figure}{Fig.}{}
\crefname{algorithm}{Algorithm}{}
\crefname{equation}{eq.}{}
\crefname{appendix}{App.}{}
\crefname{thm}{Theorem}{}
\crefname{prop}{Proposition}{}
\crefname{cor}{Corollary}{}
\crefname{observation}{Observation}{}
\crefname{assumption}{Assumption}{}
\crefformat{section}{\S#2#1#3}

\usepackage{multirow}
\title{NLP Assignment}
\begin{document}
\setserie{1}



\lectureheader{Prof. Ryan Cotterell}
{}
{\Large Natural Language Processing}{Fall 2022}
\begin{center}
    {\Huge Yannick Wattenberg: Assignment 03}\\
      \quad\newline
      ywattenberg@inf.ethz.ch, 19-947-464.\\
      \quad\newline
    \timestamp

\end{center}
\section*{Question 1: Exploring the Kleene Star}
\subsection*{a)}
To prove: 
\begin{align*}
    a^* &\stackrel{!}{=} 1 \oplus a \otimes a^* \\
\end{align*}
\begin{align}
    a^* &= \bigoplus^{\infty}_{n=0}a^{\otimes n} &&(\text{def.} *)\\
    &= 1 \oplus \bigoplus^{\infty}_{n=1}a^{\otimes n}  &&(\text{def.} \oplus)\\
    &= 1 \oplus a \otimes \bigoplus^{\infty}_{n=0}a^{\otimes n}  &&(\text{diss. of } \otimes \text{over } \oplus)\\
\end{align}

\subsection*{b)}
We want to find a Kleene Star for $\mathcal{W}_{log}$. Using the definition of Kleene Star, we have:
\begin{align}
    a^* &= \bigoplus^{\infty}_{n=0}a^{\otimes n} &&(\text{def.} *)\\
     &= 0 \oplus_{log} a \oplus_{log} a \otimes a \oplus_{log} \dots\\
     &= \ln(\exp(0) + \exp(a)) \oplus_{log} a + a \oplus_{log} \dots &&(\text{def.} \oplus, \otimes)\\
     &= \ln(\exp(\ln(\exp(0) + \exp(a))) + exp(2a)) \oplus_{log} \dots \\
     &= \ln(\exp(0) + \exp(a)+ \exp(2a)) \oplus_{log} \dots \\
     &= \ln(\sum^\infty_{n=0} \exp(a \cdot n))\\
\end{align}
For $a \geq 1$ this series will diverge as $n \to \infty $, because $\lim_{n \to \inf} \exp(a\cdot n) \geq 1$ in that case.

Thus we assume that $a < 1$ for the rest of the proof. We can then rewrite the series as a geometric series as follows: 
\begin{align}
    \sum^\infty_{n=0} \exp(a \cdot n) &= \sum^\infty_{n=0} g^n &&(\text{where } g = \exp(a))\\
    &= \frac{1}{1-g} &&(\text{limit of the geometric series})\\
    &= \frac{1}{1-\exp(a)} &&(\text{def.} g)  
\end{align}
For the Kleene Star expression we then get:
\begin{align*}
    a^* &= \ln \left( \frac{1}{1-\exp(a)} \right)
\end{align*}
\subsection*{c)}
We want to find a Kleene Star for the expectation semiring. Following the same pattern as in the previous part, the definition gives us:

\begin{align}
    \text{let} a &= \pair{x}{y} \in \mathcal{R} \times \mathcal{R} \\
    a^* &= \bigoplus^{\infty}_{n=0}a^{\otimes n}\\
    &= \bigoplus^{\infty}_{n=0}\pair{x}{y}^{\otimes n}\\
    &= \pair{1}{0} \oplus \pair{x}{y} \oplus \pair{x^2}{2xy} \oplus \pair{x^3}{3x^2y} \oplus \dots\\
    &= \bigoplus^{\infty}_{n=0}\pair{x^n}{nx^{n-1}y}\\
    &= \pair{\sum^{\infty}_{n=0} x^n}{\sum^{\infty}_{n=0} nx^{n-1}y}
\end{align}
We can now explore the limit of both of the series individually. For the first one we have:
\begin{align}
    \sum^{\infty}_{n=0} x^n &= \frac{1}{1-x} &&(\text{limit of the geometric series})\\
\end{align}
Assuming that $|x| < 1$, as otherwise the series would diverge. For the second one we have:
\begin{align}
    \sum^{\infty}_{n=0} nx^{n-1}y  
    &= y \cdot (\sum^{\infty}_{n=0} nx^{n-1}) \\
    &= y \cdot (\sum^{\infty}_{n=0} (n+1)x^{n})  \\
    &= \frac{y}{(x-1)^2} &&(\text{limit of the power series})
\end{align}
With the same assumption as before, we can now combine the two limits to get:
\begin{align*}
    \pair{x}{y}^* &= \pair{\frac{1}{1-x}}{\frac{y}{(x-1)^2}}
\end{align*}
for $|x| < 1$.

\subsection*{d)}

Now we want to find a Kleene Star for $\mathcal{W}_{lang}$. Using the definition of Kleene Star, we have:
\begin{align}
    A* &= \bigoplus^{\infty}_{n=0}A^{\otimes n} &&(\text{def.} *)\\
     &= \epsilon \cup A \cup A \otimes A \cup \dots\\ 
     &= \epsilon \cup A \cup A \circ A \cup \dots\\ 
\end{align}

\section*{Question 2: Asterating Matrices in Idempotent Semirings}
\subsection*{a)}
For the tropical semiring, let $a \in \mathcal{R}_{\geq 0}$ be an element of the tropical semiring, we get:
\begin{align}
    0 \oplus a &= \min(0, a) &&(\text{def.} \oplus)\\
    &= 0 &&(a \in \mathcal{R}_{\geq 0})\\
\end{align}

For the artic semiring, let $a \in \mathcal{R}_{\leq 0}$ be an element of the artic semiring, we get:
\begin{align}
    0 \oplus a &= \max(0, a) &&(\text{def.} \oplus)\\
    &= 0 &&(a \in \mathcal{R}_{\leq 0})\\
\end{align}
Thus we can conclude that both the tropical and artic semirings are 0-closed.

\subsection*{b)}
We prove that the $M^n$ encodes the sum of paths of length $n$ in the graph $G$ using induction over $n$.
For $n = 1$ we have:
The matrix $M$ encodes the transition matrix of the graph $G$. Thus it encodes all paths of length one.

Our induction hypothesis is that $M^n$ encodes the sum of paths of length $n$ in the graph $G$. We want to show that $M^{n+1}$ encodes the sum of paths of length $n+1$.
We have $M^{n+1} = M^{n} \otimes M$ as $M^{n}$ encodes the sum of paths of length $n$ and $M$ encodes the sum of paths of length one.
For some $i, j \in [1,...,N]$ we have:
\begin{align}
    (M^{n+1})_{i,k} = \bigoplus_{k=1}^{N} (M^n)_{ik} \otimes M_{kj}
\end{align}
Thus as we can see $(M^{n+1})_{i,k}$ encodes the sum of all paths of length $n+1$ from $i$ to $k$ as
it is the sum of all paths of length $n$ from $i$ to some $k$ concatenated with a path of length one from $k$ to $j$.

\subsection*{c)}
Assume that there is a shortest path from $i$ to $j$ that traverses more than $N-1$ transitions/edges.
It follows that at least one vertex has to be visited twice as there are only $N$ vertices.
We denote the path as follows:
\begin{align}
    i \stackrel{w_0}{\rightarrow} \underbrace{... \rightarrow}_{a_1} k \underbrace{\rightarrow ... \rightarrow}_{a_2} k \underbrace{\rightarrow ...}_{a_3} \stackrel{w_k}{\rightarrow} j
\end{align}
We can build a path of length $N-1$ from $i$ to $j$ by cutting out the path from $k$ to $k$ in the middle of the path. 

Per definition of the shortest path, we get the following for the two paths:
\begin{align}
    &w_0 \otimes a_1 \otimes a_2 \otimes a_3 \otimes w_k \oplus w_0 \otimes a_1 \otimes a_3 \otimes w_k\\
    = &(w_0 \otimes a_1) \otimes (a_2 \otimes a_3 \otimes w_k \oplus a_3 \otimes w_k)\\
    = &(w_0 \otimes a_1) \otimes (a_2 \oplus 1) \otimes (\otimes a_3 \otimes w_k)\\
    = &(w_0 \otimes a_1) \otimes (1) \otimes (\otimes a_3 \otimes w_k) &&(\text{0-closed})\\
    = &w_0 \otimes a_1 \otimes a_3 \otimes w_k
\end{align}

\subsection*{d)}
\begin{align}
    M^* &= \bigoplus^{\infty}_{n=0} M^n &&(\text{def.} *)\\
    &= \lim_{K \rightarrow \infty} \bigoplus_{n=0}{K} M^n &&(\text{def.} \oplus)\\
\end{align}
We we have shown in part b) that $M^n$ encodes the sum of paths of length $n$ in the graph $G$. Furthermore we have shown that the shortest path from $i$ to $j$ has uses at most $N-1$ transitions.
Now using the definition of the shortest path:
\begin{align}
    Z(i,j) &= \bigoplus_{\pi \in \prod(i,j)} w(\pi) \nonumber
\end{align}
We get that $\prod(i,j)$ needs only to be over all path of length $N-1$ or less (see task c)). Thus we can rewrite
the definition as follows:
\begin{align*}
    Z(i,j) &= \bigoplus_{n = 1}^{N-1} w((M^n)_{ij})
\end{align*}
As we made no further assuming on $i,j$ this holds for all $i,j \in [1,...,N]$.
We can now rewrite the Kleene start in the same manner as the previous equality holds for all $i,j \in [1,...,N]$:
\begin{align}
       \lim_{K \rightarrow \infty} \bigoplus_{n=0}^{K} M^n \\
    &= \lim_{K \rightarrow N-1} \bigoplus_{n = 0}^{K-1} w((M^n))\\
    &= \bigoplus_{n = 0}^{N-1} w((M^n))\\
\end{align}

\subsection*{e)}
\begin{algorithm}
    \SetAlgoLined
    \caption[]{Algorithm for computing $M^*$}
    \KwIn{Adjacency matrix $M$}
    \KwOut{$M^*$}
    $M' \gets I$\\
    $R  \gets M$\\
    \For(){$i = 1; i < N; i\gets i+1$}{
        $R \gets R \oplus M'$\\
        $M' \gets M' \otimes M$
    }
\end{algorithm}
The algorithm computes $M^*$ using the equality we proved in part d). It simply aggregates the sum of the $M^n$ for $n = 1,...,N-1$ in the variable $R$. The variable $M'$ is used to compute the $M^n$.
The Runtime of the algorithm is $O(N^3)$ as we need to compute $N$ matrix multiplications of size $N \times N$.

\subsection*{f)}
We want to prove that every 0-closed semiring is also idempoent.
\begin{align}
    a \oplus a\\
    = &a \otimes (1 \oplus 1)\\
    = &a \otimes 1\\
    = &a
\end{align}
Thus we have shown that every 0-closed semiring is idempoent.

\subsection*{g)}
We want to prove the following equality:
\begin{align*}
    \bigoplus_{n=0}^{K} M^n = (I \oplus M)^K
\end{align*}
We do so by induction on $K$.
The base case $K = 0$ is trivial as $M^0 = I$ thus:
\begin{align}
    I = (I \oplus M)^0 = M^0 = \bigoplus_{n=0}^{0} M^n
\end{align}
Now we assume that the equality holds for $K$ and want to show that it also holds for $K+1$:
\begin{align}
    (I \oplus M)^{K+1} &= (I \oplus M)^K \otimes (I \oplus M) \\
    &=  \bigoplus_{n=0}^{K} M^n \otimes (I \oplus M) &&(\text{distributive})\\
    &=  \bigoplus_{n=0}^{K} M^n \oplus M^{n+1} &&(\text{def.} I)\\
    &=  \bigoplus_{n=0}^{K} M^n \oplus \bigoplus_{n=1}^{K+1} M^{n} \\
    &=   M^0 \oplus \left(\bigoplus_{n=1}^{K} M^n \oplus M^{n}\right) \oplus M^{K+1} \\
    &=   M^0 \oplus \left(\bigoplus_{n=1}^{K} M^{n}\right) \oplus M^{K+1}&&(\text{Idempotent}) \\
    &=  \bigoplus_{n=0}^{K+1} M^n 
\end{align}

\subsection*{h)}
It is easy to see that the equation $M^n = \bigotimes^{\log_2n}_{k = 0} M^{\alpha_k 2^k}$ as it all natural numbers can be represented using the binary system.
More concretely we can write $\bigotimes^{\log_2n}_{k = 0} M^{\alpha_k 2^k}$ as $M^{\sum^{\log_2n}_{k = 0} \alpha_k 2^k}$ as $M^a \otimes M^b = M^{a + b}$. Then  we choose $a_k = 1 $ iff. the k-th bit of the binary representation of n is one and zero otherwise. In other words let a be the binary representation of n. Using this it becomes apparent that $\sum^{\log_2n}_{k = 0} \alpha_k 2^k = n$ as it is the way to convert the binary representation of n ($a$) to a decimal number.

A more efficient way to compute $M^* = \bigoplus^{N-1}_{n=0} M^n$ uses the following equality which holds as the semiring is 0-closed and thus idempoent (f):
\begin{align}
    \bigoplus^{N-1}_{n=0} M^n &= (I \oplus M)^{N-1} &&(\text{(g)}) \\
    (I \oplus M)^{N-1} &= \bigotimes^{\log_2 N-1}_{k = 0} (I \oplus M)^{a_k 2^k} &&(\text{(h)}) \\
\end{align}
Where $a_k$ is the k-th bit of the binary representation of $N-1$.
The algorithm is as follows:
\begin{algorithm}
    \SetAlgoLined
    \caption[]{Faster Algorithm for computing $M^*$}
    \KwIn{Adjacency matrix $M$}
    \KwOut{$M^*$}
    $M' \gets I$\\
    $R  \gets (I \oplus M)$\\
    \For(){$i = 0; i < \log_2 (N-1)+1; i\gets i+1$}{
        \If(){$a_i \text{ is } 1$}{
            $R \gets R \otimes M'$
        }
        $M' \gets M' \otimes M'$
    }
\end{algorithm}

\subsection*{i)}
In the following all norms will refer to the two norm.
First we rewrite the definition of $\lVert M \rVert$:
\begin{align}
    \lVert M \rVert &= sup_{x \neq 0} \frac{\lVert Mx \rVert}{\lVert x \rVert} \\
    &= sup_{x \neq 0; \lVert x \rVert =1}\lVert Mx \rVert
\end{align}
This holds as one can just normalize the vector $x$ to have length 1. 

Now let $U \Sigma V^T$ be the SVD of $M$. Then we have:
\begin{align}
    sup_{x \neq 0; \lVert x \rVert =1} \lVert U \Sigma V^T x\rVert\\
    &= sup_{x' \neq 0; \lVert x' \rVert =1} \lVert \Sigma x'\rVert\\
    &= \sigma_1
\end{align}
The last equality follows from the min-max theorem for singular values. Thus we have shown that $\lVert M \rVert = \sigma_1$.

\subsection*{j)}
First we reform the initial equation:
\begin{align}
    \lVert M^* - \sum_{n=0}^K A^n \rVert &= \lVert \sum_{n=0}^\infty A^n - \sum^{\infty}_{n=0} M^n \rVert\\
    &= \lVert \sum^{\infty}_{n=K+1} M^n \rVert\\
    sup_{x \neq 0; \lVert x \rVert =1} \lVert \sum^{\infty}_{n=K+1} M^n x \rVert
    &\leq \sum^{\infty}_{n=K+1} sup_{x \neq 0; \lVert x \rVert =1} \lVert M^n x \rVert \label{2j1}\\
    &\leq \sum^{\infty}_{n=K+1} \lVert M \rVert^n &&(\sigma = \sigma_{max}(A)) \label{2j2}\\
    &= \sum^{\infty}_{n=K+1} \sigma^n \\
    &= \frac{\sigma^{K+1}}{1-\sigma} \label{2j3}
\end{align}
The inequality in \eqref{2j1} holds as the one can choose the same $x$ in the left and right. The inequality in \eqref{2j2} holds as the norm is submultiplicative and the equality in \eqref{2j3} holds as we assume $\sigma < 1$ which means the series is the geometric series. This is also the condition for the sum to converge to $M^*$. If $\sigma \geq 1$ then the sum diverges therefore there is no closed form solution for the error-term.

\subsection*{k)}
\begin{align}
    \mathcal{O} \left( \frac{\sigma^{K+1}}{1-\sigma} \right) =  \mathcal{O} \left(\sigma^{K+1}\right)
\end{align}
Where $\sigma < 1$.
As this error term decayes exponentially fast we can say that the error term is acceptable and the truncation is a good approximation.
\end{document}

%%% Local Variables:
%%% mode: latex
%%% TeX-master: t
%%% End:
