\documentclass[a4paper,12pt]{ETHexercise}
\usepackage{bbm}

\textheight24cm
\topmargin-2.5cm
\oddsidemargin+0cm
\textwidth16.3cm 

\usepackage{amsmath}
\usepackage{amsfonts}
\usepackage{amssymb}
\usepackage{color}
\usepackage[latin1]{inputenc}
\usepackage{ifthen}
\usepackage{enumerate}
\usepackage{lastpage}
\usepackage{graphicx}
%\usepackage{subfigure}
\usepackage{subcaption}
\usepackage{bm}
\usepackage{tikz}
\usepackage{pgfplots}
\usepackage{float}
\usepackage{nicefrac}
\usepackage{epsfig}
\usepackage{etoolbox}
\usepackage{framed}
\usepackage{enumitem}
\usepackage{hyperref}
\usepackage{datetime}
\usepackage{url}
%\usepackage{algorithm}
\usepackage{svg}
\usepackage[ruled,vlined]{algorithm2e}
\usepackage{setspace}
\usepackage[htt]{hyphenat}
\usepackage{qtree}
\usepackage{forest}
\usepackage{inconsolata} % for texttt{}


\usepackage[noend]{algpseudocode}
\algnewcommand{\parState}[1]{\State%
    \parbox[t]{\dimexpr\linewidth-\algmargin}{\strut\hangindent=\algorithmicindent \hangafter=1 #1\strut}}

\algrenewcommand\algorithmicindent{1.0em}%
\renewcommand\algorithmicdo{:}
\renewcommand\algorithmicthen{:}
\input{math_commands}

%\newcommand{\linetofill}{\ \\\hphantom{\hspace{0mm}}\hrulefill\ \\}
\newcommand{\linetofill}{\ \\\hphantom{\hspace{0mm}}\dotfill\ \\}
\newcommand{\vlinetofill}[1]{\\\rule{#1 mm}{0.1mm}\ \\}
\newcommand{\cbox}{
  \setlength{\unitlength}{1pt}
  \begin{picture}(10,10)
    \put(0,0){\line(1,0){8}} \put(0,8){\line(1,0){8}}
    \put(0,0){\line(0,1){8}} \put(8,0){\line(0,1){8}}
  \end{picture}
}
\newcommand{\bigcbox}{
  \setlength{\unitlength}{3pt}
  \begin{picture}(10,10)(1,1)
    \put(0,0){\line(1,0){8}} \put(0,8){\line(1,0){8}}
    \put(0,0){\line(0,1){8}} \put(8,0){\line(0,1){8}}
  \end{picture}
}
% points and boxes on the right
\newcommand{\points}[1]{
\ \\[-5mm]
\hphantom{\ }\hfill\textbf{#1 pts \bigcbox \hspace{-6mm}}
\vspace{5mm}
}
% multiple choice checkboxes
%\newcommand{\boxt}{\hspace*{2em} $[~~]$ \textsf{True}}
%\newcommand{\boxf}{\hspace*{2em} $[~~]$ \textsf{False}}
\newcommand{\boxt}{\hspace*{2em} $\square$ \textsf{True}}
\newcommand{\boxf}{\hspace*{2em} $\square$ \textsf{False}}
\newcommand{\justif}{\textsf{Justification}:  \rule{0ex}{2em}\dotfill\\\rule{0ex}{2em}\dotfill}
\newcommand{\checkboxWithJustification}{\boxt \boxf\\ \justif\\}
\newcommand{\checkbox}{\boxt \boxf}
\usepackage{tikz}
\usepackage{xcolor}
\usepackage{amssymb}
\usepackage{amsmath}
\usepackage{amsthm}
\usepackage{pgfplots}
\pgfmathdeclarefunction{gauss}{2}{%
  \pgfmathparse{1/(#2*sqrt(2*pi))*exp(-((x-#1)^2)/(2*#2^2))}%
}
\renewcommand{\S}{{\cal S}}
\usepackage{pgfplots}
    \pgfplotsset{compat=newest}
    \pgfplotsset{small, every non boxed x axis/.append style={x axis line style=-},
every non boxed y axis/.append style={y axis line style=-}}

\newcommand{\timestamp}{\ddmmyyyydate\today \,\,- \currenttime h}
%%% Local Variables:
%%% mode: latex
%%% TeX-master: "exam"
%%% End:

% Numbers
\usepackage[group-separator={,}]{siunitx}

\usepackage{cleveref}
\crefname{section}{\S}{\S\S}
\Crefname{section}{\S}{\S\S}
\crefname{table}{Tab.}{}
\crefname{figure}{Fig.}{}
\crefname{algorithm}{Algorithm}{}
\crefname{equation}{eq.}{}
\crefname{appendix}{App.}{}
\crefname{thm}{Theorem}{}
\crefname{prop}{Proposition}{}
\crefname{cor}{Corollary}{}
\crefname{observation}{Observation}{}
\crefname{assumption}{Assumption}{}
\crefformat{section}{\S#2#1#3}

\usepackage{multirow}
\title{NLP Assignment}
\begin{document}
\setserie{1}



\lectureheader{Prof. Ryan Cotterell}
{}
{\Large Natural Language Processing}{Fall 2022}
\begin{center}
    {\Huge Yannick Wattenberg: Assignment 03}\\
      \quad\newline
      ywattenberg@inf.ethz.ch, 19-947-464.\\
      \quad\newline
    \timestamp

\end{center}
\section*{Question 1: Exploring the Kleene Star}
\subsection*{a)}
To prove: 
\begin{align*}
    a^* &\stackrel{!}{=} 1 \oplus a \otimes a^* \\
\end{align*}
\begin{align}
    a^* &= \bigoplus^{\infty}_{n=0}a^{\otimes n} &&(\text{def.} *)\\
    &= 1 \oplus \bigoplus^{\infty}_{n=1}a^{\otimes n}  &&(\text{def.} \oplus)\\
    &= 1 \oplus a \otimes \bigoplus^{\infty}_{n=0}a^{\otimes n}  &&(\text{diss. of } \otimes \text{over } \oplus)\\
\end{align}

\subsection*{b)}
We want to find a Kleene Star for $\mathcal{W}_{log}$. Using the definition of Kleene Star, we have:
\begin{align}
    a^* &= \bigoplus^{\infty}_{n=0}a^{\otimes n} &&(\text{def.} *)\\
     &= 0 \oplus_{log} a \oplus_{log} a \otimes a \oplus_{log} \dots\\
     &= \ln(\exp(0) + \exp(a)) \oplus_{log} a + a \oplus_{log} \dots &&(\text{def.} \oplus, \otimes)\\
     &= \ln(\exp(\ln(\exp(0) + \exp(a))) + exp(2a)) \oplus_{log} \dots \\
     &= \ln(\exp(0) + \exp(a)+ \exp(2a)) \oplus_{log} \dots \\
     &= \ln(\sum^\infty_{n=0} \exp(a \cdot n))\\
\end{align}
For $a \geq 0$ this series will diverge as $n \to \infty $, because $\lim_{n \to \inf} \exp(a\cdot n) \geq 1$ in that case.

Thus we assume that $a < 0$ for the rest of the proof. We can then rewrite the series as a geometric series as follows: 
\begin{align}
    \sum^\infty_{n=0} \exp(a \cdot n) &= \sum^\infty_{n=0} g^n &&(\text{where } g = \exp(a))\\
    &= \frac{1}{1-g} &&(\text{limit of the geometric series})\\
    &= \frac{1}{1-\exp(a)} &&(\text{def.} g)  
\end{align}
For the Kleene Star expression we then get:
\begin{align*}
    a^* &= \ln \left( \frac{1}{1-\exp(a)} \right)
\end{align*}
\subsection*{c)}
We want to find a Kleene Star for the expectation semiring. Following the same pattern as in the previous part, the definition gives us:

\begin{align}
    \text{let} a &= \pair{x}{y} \in \mathcal{R} \times \mathcal{R} \\
    a^* &= \bigoplus^{\infty}_{n=0}a^{\otimes n}\\
    &= \bigoplus^{\infty}_{n=0}\pair{x}{y}^{\otimes n}\\
    &= \pair{1}{0} \oplus \pair{x}{y} \oplus \pair{x^2}{2xy} \oplus \pair{x^3}{3x^2y} \oplus \dots\\
    &= \bigoplus^{\infty}_{n=0}\pair{x^n}{nx^{n-1}y}\\
    &= \pair{\sum^{\infty}_{n=0} x^n}{\sum^{\infty}_{n=0} nx^{n-1}y}
\end{align}
We can now explore the limit of both of the series individually. For the first one we have:
\begin{align}
    \sum^{\infty}_{n=0} x^n &= \frac{1}{1-x} &&(\text{limit of the geometric series})\\
\end{align}
Assuming that $|x| < 1$, as otherwise the series would diverge. For the second one we have:
\begin{align}
    \sum^{\infty}_{n=0} nx^{n-1}y  
    &= y \cdot (\sum^{\infty}_{n=0} nx^{n-1}) \\
    &= y \cdot (\sum^{\infty}_{n=0} (n+1)x^{n})  \\
    &= \frac{y}{(x-1)^2} &&(\text{limit of the power series})
\end{align}
With the same assumption as before, we can now combine the two limits to get:
\begin{align*}
    \pair{x}{y}^* &= \pair{\frac{1}{1-x}}{\frac{y}{(x-1)^2}}
\end{align*}
for $|x| < 1$.

\subsection*{d)}

Now we want to find a Kleene Star for $\mathcal{W}_{lang}$. Using the definition of Kleene Star, we have:
\begin{align}
    A* &= \bigoplus^{\infty}_{n=0}A^{\otimes n} &&(\text{def.} *)\\
     &= \epsilon \cup A \cup A \otimes A \cup \dots\\ 
     &= \epsilon \cup A \cup A \circ A \cup \dots\\ 
\end{align}
\end{document}

%%% Local Variables:
%%% mode: latex
%%% TeX-master: t
%%% End:
