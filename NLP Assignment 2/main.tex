\documentclass[a4paper,12pt]{ETHexercise}
\usepackage{bbm}

\textheight24cm
\topmargin-2.5cm
\oddsidemargin+0cm
\textwidth16.3cm 

\usepackage{amsmath}
\usepackage{amsfonts}
\usepackage{amssymb}
\usepackage{color}
\usepackage[latin1]{inputenc}
\usepackage{ifthen}
\usepackage{enumerate}
\usepackage{lastpage}
\usepackage{graphicx}
%\usepackage{subfigure}
\usepackage{subcaption}
\usepackage{bm}
\usepackage{tikz}
\usepackage{pgfplots}
\usepackage{float}
\usepackage{nicefrac}
\usepackage{epsfig}
\usepackage{etoolbox}
\usepackage{framed}
\usepackage{enumitem}
\usepackage{hyperref}
\usepackage{datetime}
\usepackage{url}
%\usepackage{algorithm}
\usepackage{svg}
\usepackage[ruled,vlined]{algorithm2e}
\usepackage{setspace}
\usepackage[htt]{hyphenat}
\usepackage{qtree}
\usepackage{forest}
\usepackage{inconsolata} % for texttt{}


\usepackage[noend]{algpseudocode}
\algnewcommand{\parState}[1]{\State%
    \parbox[t]{\dimexpr\linewidth-\algmargin}{\strut\hangindent=\algorithmicindent \hangafter=1 #1\strut}}

\algrenewcommand\algorithmicindent{1.0em}%
\renewcommand\algorithmicdo{:}
\renewcommand\algorithmicthen{:}
\input{math_commands}

%\newcommand{\linetofill}{\ \\\hphantom{\hspace{0mm}}\hrulefill\ \\}
\newcommand{\linetofill}{\ \\\hphantom{\hspace{0mm}}\dotfill\ \\}
\newcommand{\vlinetofill}[1]{\\\rule{#1 mm}{0.1mm}\ \\}
\newcommand{\cbox}{
  \setlength{\unitlength}{1pt}
  \begin{picture}(10,10)
    \put(0,0){\line(1,0){8}} \put(0,8){\line(1,0){8}}
    \put(0,0){\line(0,1){8}} \put(8,0){\line(0,1){8}}
  \end{picture}
}
\newcommand{\bigcbox}{
  \setlength{\unitlength}{3pt}
  \begin{picture}(10,10)(1,1)
    \put(0,0){\line(1,0){8}} \put(0,8){\line(1,0){8}}
    \put(0,0){\line(0,1){8}} \put(8,0){\line(0,1){8}}
  \end{picture}
}
% points and boxes on the right
\newcommand{\points}[1]{
\ \\[-5mm]
\hphantom{\ }\hfill\textbf{#1 pts \bigcbox \hspace{-6mm}}
\vspace{5mm}
}
% multiple choice checkboxes
%\newcommand{\boxt}{\hspace*{2em} $[~~]$ \textsf{True}}
%\newcommand{\boxf}{\hspace*{2em} $[~~]$ \textsf{False}}
\newcommand{\boxt}{\hspace*{2em} $\square$ \textsf{True}}
\newcommand{\boxf}{\hspace*{2em} $\square$ \textsf{False}}
\newcommand{\justif}{\textsf{Justification}:  \rule{0ex}{2em}\dotfill\\\rule{0ex}{2em}\dotfill}
\newcommand{\checkboxWithJustification}{\boxt \boxf\\ \justif\\}
\newcommand{\checkbox}{\boxt \boxf}
\usepackage{tikz}
\usepackage{xcolor}
\usepackage{amssymb}
\usepackage{amsmath}
\usepackage{amsthm}
\usepackage{pgfplots}
\pgfmathdeclarefunction{gauss}{2}{%
  \pgfmathparse{1/(#2*sqrt(2*pi))*exp(-((x-#1)^2)/(2*#2^2))}%
}
\renewcommand{\S}{{\cal S}}
\usepackage{pgfplots}
    \pgfplotsset{compat=newest}
    \pgfplotsset{small, every non boxed x axis/.append style={x axis line style=-},
every non boxed y axis/.append style={y axis line style=-}}

\newcommand{\timestamp}{\ddmmyyyydate\today \,\,- \currenttime h}
%%% Local Variables:
%%% mode: latex
%%% TeX-master: "exam"
%%% End:

% Numbers
\usepackage[group-separator={,}]{siunitx}

\usepackage{cleveref}
\crefname{section}{\S}{\S\S}
\Crefname{section}{\S}{\S\S}
\crefname{table}{Tab.}{}
\crefname{figure}{Fig.}{}
\crefname{algorithm}{Algorithm}{}
\crefname{equation}{eq.}{}
\crefname{appendix}{App.}{}
\crefname{thm}{Theorem}{}
\crefname{prop}{Proposition}{}
\crefname{cor}{Corollary}{}
\crefname{observation}{Observation}{}
\crefname{assumption}{Assumption}{}
\crefformat{section}{\S#2#1#3}

\usepackage{multirow}
\title{NLP Assignment}
\begin{document}
\setserie{1}



\lectureheader{Prof. Ryan Cotterell}
{}
{\Large Natural Language Processing}{Fall 2022}
\begin{center}
    {\Huge Yannick Wattenberg: Assignment 02}\\
      \quad\newline
      ywattenberg@inf.ethz.ch, 19-947-464.\\
      \quad\newline
    \timestamp

\end{center}
\section{Question: Entropy of a Conditional Random Field}
\subsection*{a)}
Prove that the \textbf{expectation semiring} satisfies the semiring axioms:

Let $x_1, y_1, x_2, y_2, x_3, y_3 \in \R$ for this subsection.
\subsubsection*{Axiom 1}
\begin{align*}
    (\R \times \R, \oplus, \pair{0}{0}) \text{ is a commutative monoid with identity element } \pair{0}{0}
\end{align*}

Associativity and Commutativity of $\oplus$:
\begin{align}
    (\pair{x_1}{y_1} \oplus \pair{x_2}{y_2}) \oplus \pair{x_3}{y_3} &\stackrel{!}{=} \pair{x_1}{y_1} \oplus (\pair{x_3}{y_3} \oplus \pair{x_2}{y_2})\\
     (\pair{x_1}{y_1} \oplus \pair{x_2}{y_2}) \oplus \pair{x_3}{y_3} &= \pair{x_1+x_2}{y_1+y_2} \oplus \pair{x_3}{y_3} &&(\text{def. } \oplus)\\
     &= \pair{(x_1+x_2)+x_3}{(y_1+y_2)+y_3} &&(\text{def. } \oplus)\\
     &= \pair{x_1+(x_2+x_3)}{y_1+(y_2+y_3)} &&(\text{ass. of } +)\\
     &= \pair{x_1}{y_1} \oplus \pair{x_2+x_3}{y_2+y_3}&&(\text{def. } \oplus)\\
     &= \pair{x_1}{y_1} \oplus \pair{x_3+x_2}{y_3+y_2}&&(\text{comm. of }+)\\
     &= \pair{x_1}{y_1} \oplus (\pair{x_3}{y_3} \oplus \pair{x_2}{y_2}) &&(\text{def. } \oplus)
\end{align}

$\pair{0}{0}$ is the identity element:
\begin{align}
    \pair{0}{0} \oplus \pair{x_1}{y_1} &\stackrel{!}{=} \pair{x_1}{y_1}\\
    \pair{0}{0} \oplus \pair{x_1}{y_1} &= \pair{x_1+0}{y_1+0} &&(\text{def. } \oplus)\\
    &=  \pair{x_1}{y_1}
    &=  \pair{x_1 + 0}{y_1 + 0}
    &= \pair{x_1}{y_1} \oplus \pair{0}{0} &&(\text{def. } \oplus)
\end{align}

\subsubsection*{Axiom 2}
\begin{align*}
    (\R \times \R, \otimes, \pair{1}{0}) \text{ is a monoid with identity element } \pair{1}{0}
\end{align*}

Associativity of $\otimes$:
\begin{align}
    &(\pair{x_1}{y_1} \otimes \pair{x_2}{y_2}) \otimes \pair{x_3}{y_3} \stackrel{!}{=} \pair{x_1}{y_1} \otimes (\pair{x_2}{y_2} \otimes \pair{x_3}{y_3})\\
    &(\pair{x_1}{y_1} \otimes \pair{x_2}{y_2}) \otimes \pair{x_3}{y_3} =
    \pair{x_1 \cdot x_2}{x_1 \cdot y_2 + y_1 \cdot x_2} \otimes \pair{x_3}{y_3} &&(\text{def. } \otimes)\\
    =  &\pair{(x_1 \cdot x_2) \cdot x_3}{(x_1 \cdot x_2) \cdot y_3 + (x_1 \cdot y_2 + y_1 \cdot x_2) \cdot x_3} &&(\text{def. } \otimes)\\
    =  &\pair{(x_1 \cdot x_2) \cdot x_3}{x_1 \cdot x_2 \cdot y_3 + x_1 \cdot y_2 \cdot x_3 + y_1 \cdot x_2 \cdot x_3 } &&(\text{diss. } \cdot)\\ 
    =  &\pair{(x_1 \cdot x_2) \cdot x_3}{x_1 \cdot (x_2 \cdot y_3 + y_2 \cdot x_3) + y_1 \cdot x_2 \cdot x_3 } &&(\text{diss. } \cdot)\\
    =  &\pair{x_1 \cdot( x_2 \cdot x_3)}{x_1 \cdot (x_2 \cdot y_3 + y_2 \cdot x_3) + y_1 \cdot (x_2 \cdot x_3) } &&(\text{ass. } \cdot)\\
    =  &\pair{x_1}{y_1} \otimes \pair{ x_2 \cdot x_3}{x_2 \cdot y_3 + y_2 \cdot x_3} &&(\text{def. } \otimes)\\
    =  &\pair{x_1}{y_1} \otimes (\pair{x_2}{y_2} \otimes \pair{x_3}{y_3}) &&(\text{def. } \otimes)
\end{align}

$\pair{1}{0}$ is the identity element:
\begin{align}
    \pair{x_1}{y_1} \otimes \pair{1}{0} &\stackrel{!}{=} \pair{x_1}{y_1}\\
    \pair{x_1}{y_1} \otimes \pair{1}{0} &= \pair{x_1 \cdot 1}{x_1 \cdot 0 + y_1 \cdot 1} &&(\text{def. } \otimes)\\
    &= \pair{x_1}{y_1}
    &= \pair{1 \cdot x_1}{1 \cdot y_1 + 0 \cdot x_1}\\
    &= \pair{1}{0} \otimes \pair{x_1}{y_1} &&(\text{def. } \otimes)
\end{align}

\subsubsection*{Axiom 3}
$\otimes$ distributes left and right over $\oplus$:
\begin{align}
        &\pair{x_1}{y_1} \otimes (\pair{x_2}{y_2} \oplus \pair{x_3}{y_3}) \stackrel{!}{=} (\pair{x_1}{y_1} \otimes \pair{x_2}{y_2}) \oplus (\pair{x_1}{y_1} \otimes\pair{x_3}{y_3})
\end{align}
\begin{align}
    &\pair{x_1}{y_1} \otimes (\pair{x_2}{y_2} \oplus \pair{x_3}{y_3}) = \pair{x_1}{y_1} \otimes \pair{x_2 + x_3}{y_2 + y_3} &&(\text{def. }\oplus)\\
    = &\pair{x_1 \cdot (x_2 + x_3)}{x_1 \cdot (x_2 + x_3) + y_1 \cdot (y_2 + y_3)} &&(\text{def. }\otimes)\\
    = &\pair{(x_1 \cdot x_2) + (x_1 \cdot x_3)}{ (x_1 \cdot y_2) + (y_1 \cdot x_2) + (x_1 \cdot y_3)  + (y_1 \cdot x_3)} &&(\text{diss. }\cdot)\\
    = &\pair{x_1 \cdot x_2}{(x_1 \cdot y_2) + (y_1 \cdot x_2)} \oplus \pair{x_1 \cdot x_3}{(x_1 \cdot y_3)  + (y_1 \cdot x_3)} &&(\text{def. }\oplus)\\
    =   &(\pair{x_1}{y_1} \otimes \pair{x_2}{y_2}) \oplus (\pair{x_1}{y_1} \otimes \pair{x_3}{y_3}) &&(\text{def. }\otimes)
\end{align}

\begin{align}
        &(\pair{x_2}{y_2} \oplus \pair{x_3}{y_3}) \otimes \pair{x_1}{y_1} \stackrel{!}{=} (\pair{x_2}{y_2} \otimes \pair{x_1}{y_1}) \oplus (\pair{x_3}{y_3} \otimes \pair{x_1}{y_1})
\end{align}

\begin{align}
    &(\pair{x_2}{y_2} \oplus \pair{x_3}{y_3}) \otimes \pair{x_1}{y_1} = (\pair{x_2 + x_3}{y_2 + y_3}) \otimes \pair{x_1}{y_1} &&(\text{def. }\oplus)\\
    = &\pair{(x_2 + x_3) \cdot x_1}{(x_2 + x_3) \cdot y_1 + (y_2 + y_3) \cdot x_1} &&(\text{def. }\otimes)\\
    = &\pair{(x_2 \cdot x_1) \cdot (x_3 \cdot x_1)}{(x_2 \cdot y_1) + (x_3 \cdot y_1) + (y_2 \cdot x_1)+ (y_3 \cdot x_1)} &&(\text{diss. }\cdot)\\
    = &\pair{x_2 \cdot x_1}{x_2 \cdot y_1 + y_2 \cdot x_1} \oplus \pair{x_3 \cdot x_1}{x_3 \cdot y_1 + y_3 \cdot x_1} &&(\text{def. }\oplus)\\
    = &(\pair{x_2}{y_2} \otimes \pair{x_1}{y_1}) \oplus (\pair{x_3}{y_3} \otimes \pair{x_1}{y_1}) &&(\text{def. }\otimes)
\end{align}

\subsubsection*{Axiom 4}
\begin{align}
    \pair{0}{0} \otimes \pair{x_1}{y_1} &\stackrel{!}{=} \pair{0}{0} \stackrel{!}{=} \pair{x_1}{y_1} \otimes \pair{0}{0}\\
    \pair{0}{0} \otimes \pair{x_1}{y_1} &= \pair{0 \cdot x_1}{0 \cdot y_1 + 0 \cdot x_1} &&(\text{def. }\otimes)\\
    &= \pair{0}{0}\\
    &= \pair{x_1 \cdot 0}{x_1 \cdot 0 + y_1 \cdot 0} \\
    &= \pair{x_1}{y_1} \otimes \pair{0}{0} &&(\text{def. }\otimes)
\end{align}
With this we have proven that the \textbf{expectation semiring} is a semiring.

\subsection*{b)}
The definition of the forward algorithm is as follows:
\begin{algorithm}
    \SetAlgoLined
    \caption{Forward algorithm}
    \label{alg:forward}
    $\beta(\textbf{w}, t_0) = \mathbbm{1}$\\
    \For{$i = 1$ \KwTo $N$}{
        $\beta(\textbf{w}, t_i) = \oplus_{t_i \in T} exp(score(t_i, t_{i+1}, \textbf{w}))\otimes \beta(\textbf{w}, t_{i-1})$
    }
\end{algorithm}
We raise this into the expectation semiring by replacing the $\oplus$ and $\otimes$ with the corresponding operations in the expectation semiring. We also replace the initialization of $\beta(w, t_0)$ with the neutral element of multiplication in the expectation semiring. Which yields the following algorithm:

\begin{algorithm}
    \SetAlgoLined
    \caption{Forward algorithm in the expectation semiring}
    \label{alg:forward_expectation}
    $\beta(\textbf{w}, t_0) = \pair{1}{0}$\\
    \For{$i = 1$ \KwTo $N$}{
        $w = exp(score(t_{i-1}, t_i, \textbf{w}))$\\
        $\beta(\textbf{w}, t_i) = \oplus_{t_i \in T} \pair{w}{-w \cdot logw}\otimes \beta(\textbf{w}, t_{i-1})$
    }
\end{algorithm}
We want to prove that the result of the forward algorithm lifted in the expectation semiring computes the unnormalized Entropy defined as:
\begin{align}
    H_u(T_\textbf{w}) &= -\sum_{t \in T^N} exp(score_\Theta(t,\textbf{w})) \cdot score_\Theta(t,\textbf{w})\\
\end{align}

We prove the statement by induction on the length of the sequence $N$. The base case is trivial, since the forward algorithm only computes the score of the first token:
\begin{align}
    \beta(\textbf{w}, t_1) &= \oplus_{t_1 \in T} \pair{w}{-w \cdot logw} \otimes \pair{1}{0}\\
    (\text{def. }\otimes) &= \oplus_{t_1 \in T} \pair{w}{-w \cdot logw} \\
    (\text{def. }w) &= \oplus_{t_1 \in T}\pair{exp(score(t_{0}, t_1, \textbf{w}))}{-exp(score(t_{0}, t_1, \textbf{w})) \cdot score(t_{0}, t_1, \textbf{w})}\\
    (\text{def. }\oplus)&= \pair{\sum_{t \in T^1}exp(score(t_{0}, t_1, \textbf{w}))}{-\sum_{t \in T^1} exp(score_\Theta(t,\textbf{w})) \cdot score_\Theta(t,\textbf{w})}
\end{align}
Our induction hypothesis is that the forward algorithm computes the unnormalized entropy for sequences of length $i$:
\begin{align}
    \beta(\textbf{w}, t_i) = \pair{\sum_{t \in T^i} exp(score_\Theta(t,\textbf{w}))}{-\sum_{t \in T^i} exp(score_\Theta(t,\textbf{w})) \cdot score_\Theta(t,\textbf{w})}
\end{align}
Induction step ($i \rightarrow i+1$):
\begin{align}
    \beta(\textbf{w}, t_{i+1}) = &\oplus_{t_{i+1} \in T} \pair{w}{-w \cdot logw} \otimes \beta(\textbf{w}, t_i)\\
    = &\oplus_{t_{i+1} \in T} \pair{w}{-w \cdot logw} \otimes \\ 
    &\pair{\sum_{t \in T^i} exp(score_\Theta(t,\textbf{w}))}{-\sum_{t \in T^i} exp(score_\Theta(t,\textbf{w})) \cdot score_\Theta(t,\textbf{w})} &&(\text{IH})\\
    = &\langle  \sum_{t_{i+1}\in T}\sum_{t \in T^i} exp(score_\Theta(t,\textbf{w}))  , \label{eq:p1}\\ 
    &\sum_{t_{i+1}\in T} w \cdot (-\sum_{t \in T^i} exp(score_\Theta(t,\textbf{w})) \cdot score_\Theta(t,\textbf{w})) \label{eq:p2}\\
    -&\sum_{t_{i+1}\in T} w \cdot log(w) \cdot (\sum_{t \in T^i} exp(score_\Theta(t,\textbf{w}))) \rangle \label{eq:p3}\\
    (\ref{eq:p1}) \Rightarrow &\sum_{t_{i+1}\in T} w \cdot \sum_{t \in T^i} exp(score_\Theta(t,\textbf{w}))\\
    = &\sum_{t_{i+1}\in T} exp(score(t_{i}, t_{i+1}, \textbf{w})) \cdot \sum_{t \in T^i} exp(score_\Theta(t,\textbf{w})) &&(\text{def. }w) \\
    = &\sum_{t \in T^i} \sum_{t_{i+1}\in T} exp(score_\Theta(t,\textbf{w}) + score(t_{i}, t_{i+1} \textbf{w}))\\
    = &\sum_{t \in T^{i+1}} exp(score_\Theta(t,\textbf{w})) \label{eq:p1r}\\
    (\ref{eq:p2}) \Rightarrow & \sum_{t_{i+1}\in T} w \cdot (-\sum_{t \in T^i} exp(score_\Theta(t,\textbf{w})) \cdot score_\Theta(t,\textbf{w}))\\
    = &\sum_{t_{i+1}\in T} exp(score(t_{i}, t_{i+1}, \textbf{w})) \cdot (-\sum_{t \in T^i} exp(score_\Theta(t,\textbf{w})) \cdot score_\Theta(t,\textbf{w})) &&(\text{def. }w)\\
    = &-\sum_{t \in T^i}\sum_{t_{i+1}\in T} exp(score_\Theta(t,\textbf{w}) + score(t_{i}, t_{i+1}, \textbf{w})) \cdot score_\Theta(t,\textbf{w}) \label{eq:p2r}\\
    (\ref{eq:p3}) \Rightarrow &\sum_{t_{i+1}\in T} w \cdot log(w) \cdot (\sum_{t \in T^i} exp(score_\Theta(t,\textbf{w})))\\
    = &\sum_{t_{i+1}\in T} exp(score(t_{i}, t_{i+1}, w)) \cdot score(t_{i}, t_{i+1},w) \cdot (\sum_{t \in T^i} exp(score_\Theta(t,\textbf{w}))) &&(\text{def. }w)\\
    = &\sum_{t \in T^i} \sum_{t_{i+1}\in T} exp(score_\Theta(t,\textbf{w}) + score(t_{i}, t_{i+1}, \textbf{w})) \cdot score(t_{i}, t_{i+1}, \textbf{w}) \label{eq:p3r}
\end{align}

\begin{align}
    (\ref{eq:p2r} - \ref{eq:p3r}) \Rightarrow -\sum_{t \in T^i} \sum_{t_{i+1}\in T} &exp(score_\Theta(t,\textbf{w}) + score(t_{i}, t_{i+1}, \textbf{w})) \cdot score_\Theta(t,\textbf{w})\\
    -\sum_{t \in T^i} \sum_{t_{i+1}\in T} &exp(score_\Theta(t,\textbf{w}) + score(t_{i}, t_{i+1}, \textbf{w})) \cdot score(t_{i}, t_{i+1}, \textbf{w})\\
    = -\sum_{t \in T^i} \sum_{t_{i+1}\in T} &exp(score_\Theta(t,\textbf{w}) + score(t_{i}, t_{i+1}, \textbf{w})) \cdot (score_\Theta(t,\textbf{w}) \cdot score(t_{i}, t_{i+1}, \textbf{w})) \\
    = -\sum_{t \in T^{i+1}} &exp(score_\Theta(t,\textbf{w})) \cdot (score_\Theta(t,\textbf{w})) \label{eq:p3and4}
\end{align}
Together (\ref{eq:p1r}) and (\ref{eq:p3and4}), we have:
\begin{align}
    \langle \sum_{t \in T^{i+1}} exp(score_\Theta(t,\textbf{w})) , \sum_{t \in T^{i+1}} exp(score_\Theta(t,\textbf{w})) \cdot score_\Theta(t,\textbf{w}) \rangle
\end{align}
Concluding the proof.
From this we can follow that
\begin{align}
    \beta(w, t_{N}) = \pair{\sum_{t \in T^N} exp(score_\Theta(t,\textbf{w}))}{-\sum_{t \in T^N} exp(score_\Theta(t,\textbf{w})) \cdot score_\Theta(t,\textbf{w})}
\end{align}
Which shows that the second part of the pair is equal to the unnormalized entropy. 
\subsection*{c)}

\end{document}

%%% Local Variables:
%%% mode: latex
%%% TeX-master: t
%%% End:
